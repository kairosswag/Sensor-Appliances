\section{Ausblick}
\subsection{Weitere Features}
\begin{description}
	\item[Einfache Einrichtung] Im Moment ist es nicht einfach die Basisstation einzurichten, vor allem die Versorgung der Station mit dem Wlan Schlüssel ist nicht intuitiv. Dafür gibt es verschiedene Lösungsmöglichkeiten. Ein Konzept das uns besonders benutzerfreundlich erscheint ist auf jeder ausgelieferte Basisstation einen QR-Code anzubringen der beim Scannen durch unsere App entsprechende Einrichtungsinformationen (Stations-Id, Wlan Schlüssel etc.) einfach und schnell über Bluetooth oder Wifi P2P austauscht.
	\item [Identifikation von Sensoren] \hfill \\
		Ist man im Besitz von mehreren Sensoren, so wird die Zuordnung der Sensoren zunehmend schwieriger. Um dem Problem entgegen zu wirken, gibt es folgende Ansätze:
		\begin{description}
			\item [Namensgebung] \hfill \\
				Den Sensoren wird jeweils ein eindeutiger Personenname zugewiesen, welcher auf dem Sensor abgedruckt ist und in der App angezeigt wird. Dies ist wesentlich Benutzerfreundlicher, da dadurch die Zuweisung im Vergleich zu Mac Adressen weniger abstrakt ist, auch wenn der Informationsgehalt der gleiche ist.
			\item [LED] \hfill \\
				Ein Auswählen der Sensoren in der App führt zu einem Blinken der LED am Sensor. Dies hat im Vergleich zur Namensgebung den Vorteil, dass man die Sensoren nicht alle absuchen muss, um den verlangten zu finden.
		\end{description}
	\item [Einsatz in Umgebungen ohne direkte Netzwerkanbindung] \hfill \\
		Die Basisstation erfordert eine direkte Anbindung an das lokale Netzwerk, ohne welche nämlich die App nicht mit der Basisstation kommunizieren kann. Ist dies nicht gegeben, z.B. in einem Keller, wo das Projekt übrigens auch ein großes Anwendungsgebiet sieht, so müssen neben Anpassungen seitens des Nutzers (Kabel verlegen, Access Point im Keller aufstellen), alternativen von unserer Seite angeboten werden:
		\begin{description}
			\item [Funk]  \hfill \\
				Die Basisstation könnte über ein proprietäres Funkmodul mit einer zweiten Basisstation in der Wohnung oder direkt mit den Sensoren kommunizieren. Das Funkmodul sollte explizit für solche Anwendungsfälle zugeschnitten sein und würde sich dementsprechend durch eine niedrigere Datenrate und Frequenz, aber höherer Stabilität bei großer Reichweite auszeichnen. 
			\item [Gerichtetes WLAN] \hfill \\
				Falls die Basisstation gerade nicht mehr in Reichweite des Access Point ist, kann auch eine Richtantenne Abhilfe schaffen, allerdings müsste diese an beiden Enden installiert werden.
		\end{description}
	\item [Verschlüsselung] \hfill \\
		Der gesamte Datenverkehr ist momentan noch unverschlüsselt und zudem nicht authentifiziert. Aufgrund der unkritischen Natur des Projekts ist eine Verschlüsselung und Authentifizierung nicht die größte Priorität. Zudem ist der Zugang standardmäßig auf das lokale Netz beschränkt. Nichtsdestotrotz sollte/muss ein ernsthaftes Produkt Gebrauch davon machen, wobei in unserem Fall, zumindest was die Verschlüsselung betrifft, eine Realisierung über HTTPS denkbar ist, da die Kommunikation ohnehin schon über HTTP abläuft.
	\item
\end{description}

\subsection{Produktion}
Für den Prototypen wurde die Basisstation als ein Raspberry Pi Modell B mit Bluetooth und WLAN Sticks umgesetzt. Für den Sensor wurde ein Sensirion Smart Gadget mit einer 3D-gedruckten Klammer verwendet. Für eine Produktion sind diese Optionen natürlich viel zu teuer. \\ 
Für die tatsächliche (Massen-)Produktion ist es vorstellbar für die Basisstation den wesentlich günstigeren PI Zero oder ein vergleichbares Board mit Linux zu verwenden. Auch ein eigenes Board mit Mikroprozessor und der benötigten Peripherie zu entwerfen stünde zur Debatte. Bei der Produktion der Basisstation wäre besonders abzuwiegen ob ein Mikroprozessor ausgewählt wird der eine Linuxdistribution betreiben kann und somit unsere, für den Prototypen implementierte, Software verwendet und weiterentwickelt werden kann, oder ob es sich lohnt die Software auf niedrigerer Ebene auf einem günstigeren Prozessor noch einmal umzusetzen. \\
Für die Sensoren ist es nötig ein eigenes Board, ähnlich dem Smart Gadget, zu entwickeln, da dies um ein vielfaches günstiger als das Smart Gadget wird und entsprechend unserer Anwendung angepasst werden sollte. Es bietet sich an die gleichen Bauteile zu benutzen. Diese Platine kann dann auch sehr klein entworfen werden um komplett in die Klammer zu passen. \\
Das Gehäuse für die Basisstation sowie die Klammer sollen für eine Produktion großer Stückzahl gespritzt werden. Beim Entwurf der endgültigen Teile sollte darauf geachtet werden, dass die Klammer relativ wasserfest sein sollte um den fehlerfreien Einsatz am Wäschestück und auch draußen zu gewährleisten. Für die Basisstation sind keine besonderen Maßnahmen nötig.
