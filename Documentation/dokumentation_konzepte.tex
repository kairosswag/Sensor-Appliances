\section{Konzepte}
Während der Entwicklung des Prototypen wurden verschiedene Konzepte erstellt um Ideen darzustellen, zu  verdeutlichen und über diese zu entscheiden.
\subsection{10+10}
Zur Ideenfindung wurde die 10+10-Methode benutzt um wichtige Entwurfsentscheidungen der Entwicklung im Voraus zu entscheiden. Im Folgenden sind Konzepte die aus diesem Prozess hervorgegangen sind zu sehen. Zuerst sollte darüber entschieden werden wie festgestellt wird wann die Wäsche trocken ist. 
\begin{figure}[htb] 
	\centerline{\includegraphics*[width=0.8\textwidth]{./10+10/Concept/01-klammer}}
	\caption{Dieses Konzept zeigt eine Wäscheklammer mit einem Sensor und entsprechenden anderen Komponenten zur Feststellung und Kommunikation des Status der Wäsche} 
	\label{10+10_klammer}
\end{figure}
\begin{figure}[htb] 
	\centerline{\includegraphics*[width=0.5\textwidth]{./10+10/Concept/02-extern}}
	\caption{Das in diesem Ansatz beschriebene Gerät soll herausfinden ob die Wäsche trocken ist indem es verschiedene Umgebungsdaten des Raumes sammelt in dem sich die Wäsche befindet}
	\label{10+10_extern}
\end{figure}
\begin{figure}[htb] 
	\centerline{\includegraphics*[width=0.5\textwidth]{./10+10/Concept/03-leine}}
	\caption{Eine Basisstation hat direkten Kontakt zu einer speziell gefertigten Wäscheleine, die dazu genutzt werden kann Feuchtigkeit über die Leitfähigkeit der Wäsche festzustellen}
	\label{10+10_leine}
\end{figure}
\begin{figure}[htb] 
	\centerline{\includegraphics*[width=0.5\textwidth]{./10+10/Concept/04-magnet}}
	\caption{Hier wird ein ähnliches Konzept wie bei der Wäscheklammer verwendet nur, dass der Befestigungsmechanismus magnetisch ist}
	\label{10+10_magnet}
\end{figure}
\begin{figure}[htb] 
	\centerline{\includegraphics*[width=0.5\textwidth]{./10+10/Concept/05-getrennt}}
	\caption{Die Leitfähigkeit soll hier wieder als ein Maß der Feuchtigkeit benutzt werden, allerdings ohne eine spezielle Leine sondern mit zwei Kontakten an der Wäsche}
	\label{10+10_getrennt}
\end{figure}
\begin{figure}[htb] 
	\centerline{\includegraphics*[width=0.5\textwidth]{./10+10/Concept/06-zeit}}
	\caption{Eine Zeitschätzung soll hier benutzt werden um festzustellen wann die gewünschte Trockenheit erreicht wurde}
	\label{10+10_zeit}
\end{figure}
\begin{figure}[htb] 
	\centerline{\includegraphics*[width=0.5\textwidth]{./10+10/Concept/07-zug}}
	\caption{Wie viel Wasser die Wäsche noch enthält soll über das Gewicht und somit über den Zug auf die Leine festgestellt werden}
	\label{10+10_zug}
\end{figure}
\FloatBarrier
Am Ende haben wir uns für die Wäscheklammer mit Sensor entschieden. Dieser Ansatz hat den Vorteil, dass Nutzer keine zusätzlichen Leinen kaufen müssen und er auch mit z.B. einem Wäscheständer verwendbar ist. Eine genaue Messung der Feuchtigkeit soll die direkte Anbringung am Kleidungsstück ermöglichen, die Klammer erfüllt außerdem die gleiche Funktion wie eine normale Wäscheklammer.
\\ Im Weiteren 